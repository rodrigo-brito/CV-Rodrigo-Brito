%------------------------------------------------
% Original Template by Alessandro Plasmati
% Email: alessandro.plasmati@gmail.com
% Edited by: Rodrigo Brito (www.rodrigobrito.net)
%------------------------------------------------
\documentclass[a4paper,10pt]{article}
\usepackage[margin=1in]{geometry}
\addtolength{\voffset}{0cm}
\usepackage{marvosym}
\usepackage{fontspec}
\usepackage{xunicode,xltxtra,url,parskip}
\RequirePackage{color,graphicx}
\usepackage[usenames,dvipsnames]{xcolor}
\usepackage[big]{layaureo}
\usepackage{supertabular}
\usepackage{titlesec}

\usepackage{hyperref}
\definecolor{linkcolour}{rgb}{0,0.2,0.6}
\hypersetup{colorlinks,breaklinks,urlcolor=linkcolour, linkcolor=linkcolour}

%FONTS (OpenSans)
\defaultfontfeatures{Mapping=tex-text}
\setmainfont[
SmallCapsFont =regular.ttf,
BoldFont = bold.ttf,
ItalicFont = italic.ttf
]
{regular.ttf}

\titleformat{\section}{\Large\scshape\raggedright}{}{0em}{}[\titlerule]
\titlespacing{\section}{0pt}{2pt}{2pt}

\usepackage[absolute]{textpos}

\setlength{\TPHorizModule}{20mm}
\setlength{\TPVertModule}{\TPHorizModule}
\textblockorigin{2mm}{0.65\paperheight}
\setlength{\parindent}{0pt}

%--------------------BEGIN DOCUMENT----------------------
\begin{document}

\pagestyle{empty} % non-numbered pages

\font\fb=''[cmr10]'' %for use with \LaTeX command

\vspace*{-1cm}
%--------------------TITLE-------------
\par{\centering
	{\Huge Rodrigo Ferreira de Brito
}\bigskip\par}

%--------------------SECTIONS-----------------------------------
\section{Dados Pessoais}

\begin{tabular}{rl}
Idade & 23 anos\\
Endereço & Rua Ipê Roxo, 73, Vila Francisco de Moura - Sabará/MG\\    
Telefones & (31) 3672-6290 / (31) 98377-7950\\
E-mail & contato@rodrigobrito.net\\
Portfólio & www.rodrigobrito.net\\
\end{tabular}

\section{Objetivo Profissional}

Atuar como Desenvolvedor Web ou Mobile Junior.

\section{Formação}
\begin{tabular}{r|p{13.1cm}}
	\textsc{Dez 2016} & Bacharelado em Sistemas de Informação - 8º Período\\
	\textsc{Jan 2013} & \textbf{Instituto Federal de Minas Gerais - IFMG}\\
	&\small\emph{Previsão de formatura: Dezembro/2016}\\
	\multicolumn{2}{c}{} \\
	\textsc{Jun 2011}& Curso Técnico em Informática \\
	\textsc{Jan 2010}& \textbf{Colégio Augustus}
\end{tabular}

\section{Experiência}
\begin{tabular}{r|p{13.1cm}}
	\emph{Atual} & Estágio em Desenvolvimento Web \\
	\textsc{Ago 2015}&\emph{\textbf{Prefeitura Municipal de Sabará}}\\
		 &\footnotesize{Análise e desenvolvimento do Sistema Integrado da Guarda Municipal de Sabará (SIGMA). As atividades envolvem levantamento de requisitos, desenvolvimento de software e testes. Atuação com tecnologias WEB como PHP, MySQL, HTML5, CSS3, Angular JS, Laravel, Git e gerência de servidores Linux.}\\
	\multicolumn{2}{c}{} \\
	\textsc{Mai 2013} & Instrutor Freelancer de Informática \\
		 \textsc{Jun 2011}&\emph{\textbf{Microlins}}\\
		 &\footnotesize{Responsável por ministrar as disciplinas Informática Básica, Montagem e Manutenção e Rede Estruturada.}\\
	\multicolumn{2}{c}{} \\
	\textsc{Fev 2011} & Estágio em desenvolvimento Java\\
		\textsc{Ago 2010}&\emph{\textbf{PF\&J Automação Bancária}}\\
		& \footnotesize{Desenvolvimento de aplicações Java para Desktop, atuação com tecnologias como Hibernate, C3P0, Mysql, entre outras.}
\end{tabular}

\section{Formação Complementar}
Idioma, Inglês - Luziana Lanna, \textsc{2014} (108 horas).\\
Profissionalizante, Rotinas Administrativas - Microlins, \textsc{2010} (72 horas).\\
Programa de Iniciação Científica em Matemática - UFMG, 2010.\\
Programa de Iniciação Científica em Matemática - UFMG, 2009.\\
Curso de Comunicação e Jornalismo - Conselho Municipal da Juventude de Sabará, 2007 (18 horas).

\section{Conhecimento Técnico}
\begin{tabular}{rl}
	Softwares: & Photoshop, Ilustrator, Eclipse, Visual Studio, Sublime, Git.\\
	Linguagens: & Java EE, Android, PHP, C/C++, NodeJS, Python, SQL, HTML5, CSS3, JS.\\
	Frameworks/Libs: & Laravel, WordPress, QT, Hibernate, Spring, Angular JS, jQuery, Phonegap, Ionic.\\
	Utilitários: & Gulp, Grunt, Bower, NPM, Composer, SASS/LESS.
\end{tabular}

%\section{Publicações}
%\small
%BRITO, R. F.; GOMES, B. N.; SILVA, A. A. F. Um estudo de caso abordando o problema de localização de concentradores com alocação simples em um estado brasileiro. \textbf{XXXVI Iberian Latin American Congress on Computational Methods in Engineering}. 2015, Rio de Janeiro.\\ \\
%BRITO, R. F.; GOMES, B. N.; CAMARGO, R. S. Uma eficiente heurística para o projeto de redes eixo-raio: um estudo de caso para as cidades de Minas Gerais. \textbf{XLVII Simpósio Brasileiro de Pesquisa Operacional}. 2015, Porto de Galinhas.
\end{document}
